\begin{problem}{Introduction to Time Complexity }{standard input}{standard output}{1 second}{256 megabytes}

\textbf{This problem will not be graded for Homework 1} 

In algorithmic studies, we often focus on understanding how an algorithm works and analyzing its theoretical time complexity. However, in real-world applications, what really matters is the actual size of the data and the time constraints within which the algorithm must execute.

A widely accepted rule of thumb for estimating execution time, particularly on personal computers and online coding platforms, is that approximately $10^8$ operations take around 1 second.

So, if an algorithm runs in $O(n)$ time, and $n \le 10^8$, you can generally expect it to complete in under 1 second. However, this estimate can vary depending on the programming language and the specific hardware running the code.

Given an array of length $n$, your task is to find and output the maximum sum of any contiguous subarray within it. Note that the empty subarray is considered and has a sum of 0. Once you've implemented your solution, evaluate its time complexity to ensure it meets the real-world time constraints mentioned above.

\textbf{Hint: } For this problem, your grade will depend on the efficiency of your algorithm. If you implement an $O(n^3)$ algorithm, you will receive 60\%. If you manage to optimize it to $O(n^2)$, you'll receive 80\%. Achieving a solution with $O(n)$ time complexity will earn you 100\%. However, this problem is ungraded, so feel free to experiment and try out different approaches without worrying about the final grade.


\InputFile
The first line contains a positive integer $n$

The second line contains $n$ integers $a_1, \cdots, a_n$. For all $i$, $-100 \leq a_i \leq 100$.

\OutputFile
Output a single integer, indicating the maximum sum. Note that empty subarrays are valid and they have a sum of 0.

\Scoring
There are some subtasks in this problem, you will get the percentage of score if you pass the subtask

\begin{center}
  \begin{tabular}{ | c | c | c | c | } \hline
    \bf{Subtask} &
    \bf{Condition} &
    \bf{Score} &
    \bf{Additional Limitations} \\ \hline
    $1$ & $n \le 10$ & $20\%$ & None \\ \hline
    $2$ & $n \le 100$ & $20\%$ & Must pass Subtask 1 \\ \hline
    $3$ & $n \le 1000$ & $20\%$ & Must pass Subtask 1, 2 \\ \hline
    $4$ & $n \le 10^4$ & $20\%$ & Must pass Subtask 1, 2, 3 \\ \hline
    $5$ & $n \le 10^6$ & $20\%$ & Must pass Subtask 1, 2, 3, 4 \\ \hline
    \end{tabular}
\end{center}

\Example

\begin{example}
\exmpfile{example.01}{example.01.a}%
\end{example}

\Note
The subarray [3, -2, 5, 3] gives the maximum sum, which is 9. 

\end{problem}

