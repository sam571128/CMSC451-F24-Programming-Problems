\textbf{This problem will not be graded for Homework 1} 

In algorithmic studies, we often focus on understanding how an algorithm works and analyzing its theoretical time complexity. However, in real-world applications, what really matters is the actual size of the data and the time constraints within which the algorithm must execute.

A widely accepted rule of thumb for estimating execution time, particularly on personal computers and online coding platforms, is that approximately $10^8$ operations take around 1 second.

So, if an algorithm runs in $O(n)$ time, and $n \le 10^8$, you can generally expect it to complete in under 1 second. However, this estimate can vary depending on the programming language and the specific hardware running the code.

Given an array of length $n$, your task is to find and output the maximum sum of any contiguous subarray within it. Note that the empty subarray is considered and has a sum of 0. Once you've implemented your solution, evaluate its time complexity to ensure it meets the real-world time constraints mentioned above.

\textbf{Hint: } For this problem, your grade will depend on the efficiency of your algorithm. If you implement an $O(n^3)$ algorithm, you will receive 60\%. If you manage to optimize it to $O(n^2)$, you'll receive 80\%. Achieving a solution with $O(n)$ time complexity will earn you 100\%. However, this problem is ungraded, so feel free to experiment and try out different approaches without worrying about the final grade.
